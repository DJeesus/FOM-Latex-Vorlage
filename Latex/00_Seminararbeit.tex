\documentclass[a4paper,12pt]{scrreprt}%Dokumentklasse
\usepackage[left=4cm, right=2cm, top=2.8cm, bottom=2cm,
	includehead=true, % Kopfzeile innerhalb des Textkörper, also nicht im Rand
    includefoot=false, % muss true gesetzt werden, wenn eine Fußzeile genutzt wird!!
    %showframe,%Anzeigen der Frames für das Messen
    bindingoffset=0mm]{geometry}
    
\renewcommand{\footnotesize}{\fontsize{10}{10} \selectfont}%schriftgröße der Fußziele  
    
\usepackage[onehalfspacing]{setspace}
% ============= Packages =============

% Dokumentinformationen
\usepackage[
	pdftitle={Analyse von Erhebungstechniken},
	pdfsubject={},
	pdfauthor={Ingo Sewcz},
	pdfkeywords={Erhebungstechniken, Dokumentenanalyse, Laufzettel, Interview},	
	%Links nicht einrahmen
	hidelinks
]{hyperref}


% Standard Packages
\usepackage[utf8]{inputenc}
\usepackage[ngerman]{babel}
\usepackage[T1]{fontenc}
\usepackage{graphicx, subfig}
\graphicspath{{img/}}
\usepackage[
  automark,
  autooneside=false,% <- needed if you want to use \leftmark and \rightmark in a onesided document
  headsepline
]{scrlayer-scrpage}
\usepackage{lmodern}
\usepackage{color}


% zusätzliche Schriftzeichen der American Mathematical Society
\usepackage{amsfonts}
\usepackage{amsmath}

%nicht einrücken nach Absatz
\setlength{\parindent}{0pt}
%Aber Abstand zwischen Absätzen
\setlength{\parskip}{12pt}


%Zitation
\usepackage[backend=biber, style=authoryear]{biblatex}
\usepackage[babel, german=quotes]{csquotes}
\setlength{\bibitemsep}{12pt}
\addbibresource{Literatur.bib}
%Zietert wird dann an der gewünschten Stelle wie folgt:
%\parencite[vgl.][S.60]{mueller2000}
%\cite[vgl.][S.60]{mueller2000}

% ============= Kopf- und Fußzeile =============
\clearpairofpagestyles
\ihead{\leftmark}%LINKS
\chead{}%MITTE
\ohead{\thepage}%RECHTS
\renewcommand*{\headfont}{\normalfont}
\pagestyle{scrheadings}
\renewcommand*{\chapterpagestyle}{scrheadings} 

\RedeclareSectionCommands[
  beforeskip=-.5\baselineskip,
  afterskip=.25\baselineskip
]{chapter,section,subsection,subsubsection}
\RedeclareSectionCommands[
  beforeskip=.5\baselineskip,
  afterskip=-1em]{paragraph,subparagraph}

% ============= Package Einstellungen & Sonstiges ============= 
%Besondere Trennungen
\hyphenation{De-zi-mal-tren-nung}


% ============= Dokumentbeginn =============

\begin{document}
%Seiten ohne Kopf- und Fußzeile sowie Seitenzahl
\pagestyle{empty}
%Seitennummerierung auf Römisch umstellen
\pagenumbering{Roman}

\begin{center}
\begin{tabular}{p{410pt}}%die Breite muss hier gesetzt sein, da sonst eine Overflow Warning in Latex auftritt.


\begin{center}
\includegraphics[scale=0.4]{img/fomlogo.png}
\end{center}


\\

\begin{center}
\LARGE{\textsc{
Drei Erhebungstechniken:\\Dokumentenanalyse,\\ Laufzettel und Interview\\
}}
\end{center}

\\

\begin{center}
\textbf{\Large{Scientific Essay}}
\end{center}


\begin{center}
im Fach\\
Geschäftsprozessmodellierung \\
des Studiengangs Wirtschaftsinformatik - Business Information Systems \\
(Fachsemester IV)
\end{center}


\begin{center}
vorgelegt von
\end{center}

\begin{center}
\large{\textbf{Ingo Sewcz}} \\
\small{geboren am 24.07.1994 in Marl}
\end{center}

\begin{center}
\large{am 31.05.2020, in Münster}
\end{center}

\\

\begin{center}
\begin{tabular}{lll}
\textbf{Erstprüfer:} & & Prof. Dr.-Ing. Volker Engels\\
\end{tabular}
\end{center}

\end{tabular}
\end{center}


\include{02_danksagungen_sperrvermerk}

\include{03_zusammenfassung}

% Beendet eine Seite und erzwingt auf den nachfolgenden Seiten die Ausgabe aller Gleitobjekte (z.B. Abbildungen), die bislang definiert, aber noch nicht ausgegeben wurden. Dieser Befehl fügt, falls nötig, eine leere Seite ein, sodaß die nächste Seite nach den Gleitobjekten eine ungerade Seitennummer hat. 
\cleardoubleoddpage

% pagestyle für gesamtes Dokument aktivieren
\pagestyle{scrheadings}

%Inhaltsverzeichnis
\tableofcontents

%Verzeichnis aller Bilder
\newpage
\addcontentsline{toc}{section}{Abbildungsverzeichnis}
\listoffigures

%Verzeichnis aller Tabellen
\newpage
\addcontentsline{toc}{section}{Tabellenverzeichnis}
\listoftables

\newpage
%Seitennummerierung auf Arabisch umstellen
\pagenumbering{arabic}

\include{04_einleitung}

\include{05_grundlagen}

\include{06_standdertechnik}

\include{07_methoden}

\include{08_ergebnisse}

\chapter{Diskussion}
\label{sec:diskussion}

\section{Zusammenfassende Bewertung}
\label{sec:bewertung}

\section{Ausblick}
\label{sec:ausblick}

%Literaturverzeichnis
\newpage
%\bibliographystyle{unsrt}
%\bibliography{Literatur}
%Ersetzt duch:
\addcontentsline{toc}{section}{Literatur}
\printbibliography 

\newpage
%Seiten ohne Kopf- und Fußzeile sowie Seitenzahl
\pagestyle{empty}
\include{10_eidesstattlicheErklaerung}

\end{document}
